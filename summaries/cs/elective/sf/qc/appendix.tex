\chapter{Formulary}
\section{Gates}
	\begin{table}[H]
		\centering
		\begin{tabular}{ccc}
			\toprule
			\textbf{Name}    & \textbf{Matrix}                                                                                                      & \textbf{Circuit} \\ \midrule
			Hadamard         & \( H = \frac{1}{\sqrt{2}} \begin{bmatrix} 1 & 1 \\ 1 & -1 \end{bmatrix} \)                                           & \( \begin{aligned} \Qcircuit @C=1em @R=1em { & \gate{H} & \qw } \end{aligned} \) \\
			Pauli-X          & \( X = \begin{bmatrix} 0 & 1 \\ 1 & 0 \end{bmatrix} \)                                                               & \( \begin{aligned} \Qcircuit @C=1em @R=1em { & \gate{X} & \qw } \end{aligned} \) \\
			Pauli-Y          & \( Y = \begin{bmatrix} 0 & -i \\ i & 0 \end{bmatrix} \)                                                              & \( \begin{aligned} \Qcircuit @C=1em @R=1em { & \gate{Y} & \qw } \end{aligned} \) \\
			Pauli-Z          & \( Z = \begin{bmatrix} 1 & 0 \\ 0 & -1 \end{bmatrix} \)                                                              & \( \begin{aligned} \Qcircuit @C=1em @R=1em { & \gate{Z} & \qw } \end{aligned} \) \\
			Phase            & \( S = \begin{bmatrix} 1 & 0 \\ 0 & i \end{bmatrix} \)                                                               & \( \begin{aligned} \Qcircuit @C=1em @R=1em { & \gate{S} & \qw } \end{aligned} \) \\
			\(\pi/8\)        & \( T = \begin{bmatrix} 1 & 0 \\ 0 & e^{i \pi / 4} \end{bmatrix} \)                                                   & \( \begin{aligned} \Qcircuit @C=1em @R=1em { & \gate{T} & \qw } \end{aligned} \) \\
			Controlled-NOT   & \( \CNOT = \begin{bmatrix} \ID & 0 \\ 0 & X \end{bmatrix} \)                                                         & \( \begin{aligned} \Qcircuit @C=1em @R=1em { & \ctrl{1} & \qw \\ & \targ & \qw } \end{aligned} \) \\
			SWAP             & \( \mathit{SWAP} = \begin{bmatrix} 1 & 0 & 0 & 0 \\ 0 & 0 & 1 & 0 \\ 0 & 1 & 0 & 0 \\ 0 & 0 & 0 & 1 \end{bmatrix} \) & \( \begin{aligned} \Qcircuit @C=1em @R=1.75em { & \qswap & \qw \\ & \qswap \qwx & \qw } \end{aligned} \) \\
			Controlled-Z     & \( \CZ = \begin{bmatrix} \ID & 0 \\ 0 & Z \end{bmatrix} \)                                                           & \( \begin{aligned} \Qcircuit @C=1em @R=1em { & \ctrl{1} & \qw \\ & \gate{Z} & \qw } \end{aligned} \) \\
			Controlled-Phase & \( \mathit{CS} = \begin{bmatrix} \ID & 0 \\ 0 & S \end{bmatrix} \)                                                   & \( \begin{aligned} \Qcircuit @C=1em @R=1em { & \ctrl{1} & \qw \\ & \gate{S} & \qw } \end{aligned} \) \\
			Toffoli          & \( \mathit{TOFFOLI} = \begin{bmatrix} \ID & 0 & 0 \\ 0 & \ID & 0 \\ 0 & 0 & X \end{bmatrix} \)                       & \( \begin{aligned} \Qcircuit @C=1em @R=1em { & \ctrl{2} & \qw \\ & \ctrl{1} & \qw \\ & \targ & \qw } \end{aligned} \) \\
			\bottomrule
		\end{tabular}
		\caption{Essential Quantum Gates and Circuit Symbols}
	\end{table}
% end

\section{Gate Equivalences}
	\begin{itemize}
		\item \( H X H = Z \)
		\item \( H Z H = X \)
		\item \( XZ \equiv ZX \equiv Y \)
		\item \(
			\begin{aligned}
				\Qcircuit @C=1em @R=1em {
				 & \qw      & \ctrl{1} & \qw      & \qw \\
				 & \ctrl{1} & \targ    & \ctrl{1} & \qw \\
				 & \targ    & \qw      & \targ    & \qw
				}
			\end{aligned}
			=
			\begin{aligned}
				\Qcircuit @C=1em @R=1em {
				 & \ctrl{1} & \ctrl{2} & \qw \\
				 & \targ    & \qw      & \qw \\
				 & \qw      & \targ    & \qw
				}
			\end{aligned}
			\)
		\item \(
			\begin{aligned}
				\Qcircuit @C=1em @R=1em {
				 & \qw      & \ctrl{1} & \ctrl{2} & \qw \\
				 & \ctrl{1} & \targ    & \qw      & \qw \\
				 & \targ    & \qw      & \targ    & \qw
				}
			\end{aligned}
			=
			\begin{aligned}
				\Qcircuit @C=1em @R=1em {
				 & \ctrl{1} & \qw      & \qw \\
				 & \targ    & \ctrl{1} & \qw \\
				 & \qw      & \targ    & \qw
				}
			\end{aligned}
			\)
	\end{itemize}
% end

\section{Miscellaneous}
\subparagraph{Computational to Hadamard Basis}
\begin{gather}
	c_+ \ket{+} + c_- \ket{-}
	= \frac{1}{\sqrt{2}} \bigl( c_+ (\ket{0} + \ket{1}) + c_- (\ket{0} - \ket{1}) \bigr)
	= \frac{c_+ + c_-}{\sqrt{2}} \ket{0} + \frac{c_+ - c_-}{\sqrt{2}} \ket{1} \\
\end{gather}

\subparagraph{Diagonalization:}
\begin{equation}
	\mat{A} = \mat{S} \mat{D} \mat{S}^{-1},\qquad
	\mat{D} = \diag(\lambda_1, \lambda_2, \dots, \lambda_n),\qquad
	\mat{S} = \begin{bmatrix} \vec{v}_1 & \vec{v}_2 & \cdots & \vec{v}_n \end{bmatrix}
\end{equation}

\subparagraph{Geometric Series:}
	\begin{equation}
		\sum_{k = 0}^{n} x^k = \frac{1 - x^{n + 1}}{1 - x}
	\end{equation}

	\subparagraph{Number Divided by its Square-Root}
		\begin{equation}
			\frac{x}{\sqrt{x}} = \sqrt{x}
		\end{equation}
	% end
% end
